\documentclass[a4paper, titlepage]{jsarticle}

\date{\today}
\usepackage[dvipdfmx]{graphicx}
\usepackage{url}
% \usepackage[T1]{fontenc}
\usepackage{float}
\usepackage{ascmac}
\usepackage{pdfpages}

\title{ドローン宅配事業者支援システム}

\author{土佐山田IT株式会社 \and
        久保田 天治 \and 塩澤 康志 \and 蝉 祐介 \and 寺内 俊輔 \and 林 晃太郎 \and 松本 吏司}

\begin{document}
\maketitle

\tableofcontents

\clearpage

\section{現状の課題}
2022年日本における宅配便取扱個数は50億個を超えた。
\begin{figure}[H]
    \centering
    \includegraphics[width=0.6\textwidth]{./home_delivery.pdf}
    \caption{ドローン宅配}
    \label{fig:delivery}
\end{figure}
毎年増加が続いている数字からも分かるように、
現代社会において宅配サービスは日常生活に深く根付いている、しかし宅配需要の増加と少子高齢化社会に伴う労働人口の減少により、現状の宅配サービスを維持することは日々困難になっている。
\begin{figure}[H]
    \centering
    \includegraphics[width=0.8\textwidth]{./working_population.pdf}
    \caption{ドローン宅配}
    \label{fig:delivery}
\end{figure}
特に過疎地域等においては、現状の輸送方法では効率が悪く輸送方法の効率化が求められている。

\section{課題の解決方法}
上記の課題を解決するため、効率的かつ人手不足を解消できるようなドローン宅配宅配事業者支援システムを提案する。
このシステムによる課題の解決方法を以下に示す。
\begin{itemize}
    \item ドローンの操縦を自動化することでより少ない人手で大量の荷物を個人宅に運搬できるようにする。
    \item 宅配事業者の支援システムを多くの事業者に提供することで、より多くの事業者に参入してもらい、競争が生まれることでドローン宅配事業の効率化を進めることができる。
    \item 多くの事業者に共通のシステムを使ってもらうことで、スケールメリットを活かしてコスト削減を進めることが出来る。
    \item 離島や里山の陸路では宅配が大変な地域にも簡単に配送を行うことができる。
\end{itemize}

\section{機能の概要・前提条件・制約事項}
\subsection{機能概要}
本システムは管理者,ドローン宅配事業者,使用者が存在する.それぞれに向けた機能の概要を説明する.
\subsubsection{管理者向け機能}
\begin{itemize}
    \item ログインログアウト機能
    \item 会員管理機能
    \item ドローン宅配事業者会員登録承認機能
    \item 会員情報閲覧機能
    \item 利用情報分析機能
    \item 宅配依頼受付機能
    \item 宅配仕事割り振り機能
    % \item ドローン貸出機能
    % \item 貸出ドローン情報管理機能
\end{itemize}
\subsubsection{ドローン宅配事業者向け機能}
\begin{itemize}
    \item ログインログアウト機能
    \item ドローン宅配事業者会員登録申請
    \item 宅配依頼承諾機能
    \item 配達完了通知機能
    \item 宅配場所登録機能
    %\item ドローンレンタル機能
    \item 退会機能
\end{itemize}
\subsubsection{使用者向け機能}
\begin{itemize}
    \item ログインログアウト機能
    \item 使用者会員登録機能
    \item 宅配依頼機能
    \item 宅配場所登録機能
    \item 退会機能

\subsection{前提条件}
\begin{itemize}
	\item 宅配元と宅配先がお互いドローン着地地点を登録している
	\item 事業者は搬送装置を無人航空機の登録制度に基づき国土交通省のシステムに登録すること
	\item 利用者は日本語を理解できること
\end{itemize}
\subsection{制約条件}
\begin{itemize}
	\item 利用者が情報を登録するときに虚偽の情報を登録しないこと
	\item 管理者は利用者の情報を本システムの利用目的以外で使用しないこと
	\item 管理者は利用者の個人情報の管理を情報漏洩のないように行うこと
	\item 事業者は航空法施行規則236条に基づき搬送装置の適切な運用を行うこと
	\item 利用者は利用規約を順守すること
\end{itemize}
\section{情報・金銭の流れ}

\section{想定する利用者}
本システムが想定する利用者を下記に示す.
\begin{itemize}
        \item 宅配を希望する小規模事業者または個人
        \item 薬を処方する病院または診療所
\end{itemize}

\section{運用・保守}

\section{ハードウェア・ソフトウェアの構成}
\subsection{ハードウェア構成}
本システムのハードウェア構成を表\ref{fig:hardware}に示す.
\begin{table}[H]
 \begin{center}
  \caption{ハードウェアの構成}
    \label{fig:hardware}
  \begin{tabular}{ccc} \hline
    項目 & 種類 & 数量 \\ \hline \hline
    メインサーバ & レンタルサーバ & 2 \\
    データベースサーバ & レンタルサーバ & 2 \\
    管理者端末 & PC & 1 \\
    事業者端末 & PC & 事業者数 \\
    利用者端末 & スマートフォン & 利用者数 \\
    搬送装置 & ドローン & 事業者数以上 \\ \hline
  \end{tabular}
 \end{center}
\end{table}
\subsection{ソフトウェア構成}
本システムのソフトウェア構成を表\ref{fig:software}に示す.
\begin{table}[H]
 \begin{center}
  \caption{ソフトウェアの構成}
    \label{fig:software}
  \begin{tabular}{cc} \hline
    項目 & ソフトウェア \\ \hline \hline
    Webサーバ & 未定 \\
    データベース管理システム & 未定 \\
    管理者端末 & Linux系OS \\
    事業者端末 & 未定 \\
    利用者端末 & Android,iOS \\
    搬送装置 & 未定 \\ \hline
  \end{tabular}
 \end{center}
\end{table}

\section{費用・効果}
\subsection{システムの効果}
本システムを導入することによって,ドローンを用いた無人宅配が可能となり,宅配サービスの人手不足を解消することができると考えられる.また,これから先進するであろうリモート診療に利用できると考えられる.

\subsection{収益}
本システムの収益は配達による収入を想定している.配送料を600円,1日に配達する荷物の数を20,000個とすると5年間の配達による収入は
\begin{center}
    600円×20,000個×30日×60ヶ月=21,600,000,000円
\end{center}
となる.

\subsection{システムの導入・運用コスト}
本システムの導入コストは表1,運用コストは表2のようになる.
\begin{table}[htbp]
    \centering
    \begin{tabular}{c c c c c}
    \hline
    項目 & 単価(円) & 数量 & 金額(円) & 備考 \\
    \hline \hline
    物流ドローン & 3,000,000 & 200台 & 600,000,000 &  \\
    サーバ(レンタル) & 2,000 & 4ヶ月×2台 & 16,000 & \\
    \hline \hline
     & 合計 &  & 600,016,000 &  \\
    \hline
    \end{tabular}
    \caption{導入コスト}
    \label{tab:label1}
\end{table}

\begin{table}[htbp]
    \centering
    \begin{tabular}{c c c c c}
    \hline
    項目 & 単価(円) & 数量 & 金額(円) & 備考 \\
    \hline \hline
    サーバ(レンタル) & 2,000 & 60ヶ月×2台 & 240,000 & 12ヶ月×5年×1台 \\
    維持費用 & 600,016,000 & 5年 & 300,008,000 & (導入コスト×10\% )×5年 \\
    \hline \hline
     & 合計 &  & 300,248,000 &  \\
    \hline
    \end{tabular}
    \caption{運用コスト}
    \label{tab:label2}
\end{table}

上記を踏まえて,システムの開発と5年間の運用にかかる費用は次のようになる.
\begin{center}
    600,016,000円(導入コスト)+300,248,000円(運用コスト)=900,264,000円
\end{center}

\subsection{利益}
本システムを5年間運用した際の利益は以下のようになる.
\begin{center}
    21,600,000,000円(収益)-900,264,000(費用)=20,699,736,000円
\end{center}

\section{開発体制と工程計画}
本システムの開発は土佐山田IT株式会社の8名により実装する.

本システムの工程計画は図\ref{schedule}に示す.
\begin{figure}[htbp]
        \label{schedule}
        \centering
        \includegraphics[width=15cm]{schedule.png}
        \caption{工程計画}
\end{figure}

\section{本システムのアピールポイント}
\begin{enumerate}
        \item 未だに日本に存在のしないシステムである.
        \item 22年度の宅配荷物数は50億個であり,8年連続史上最高を更新中である.しかし,人手不足が深刻化しており,この課題の解決策のひとつとなりうる.
        \item 交通渋滞や地理的な制約を受けずに荷物を迅速かつ効率的に配達することが可能である.
        \item 災害時などで交通インフラが破壊されたとしても物理的障害の影響が少ないため,必要な物資や医療用品の迅速な配達が可能である.
        \item 離島や山間部などのアクセスが難しい地域において従来の宅配方法と比較し,大幅なコスト削減になる.
        \item 今日リモート診療が普及し始めている.リモート診療とドローン宅配を併用すると患者は外出する必要がなくなるため,感染リスクを極力減らすことができる.
\end{enumerate}

\section{貢献度}


\end{document}
