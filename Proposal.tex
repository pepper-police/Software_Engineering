\documentclass[a4paper, titlepage]{jsarticle}

\date{\today}
\usepackage[dvipdfmx]{graphicx}
\usepackage{url}
\usepackage[T1]{fontenc}
\usepackage{float}
\usepackage{ascmac}

\title{ドローン宅配事業者支援システム}

\author{土佐山田IT株式会社}

\begin{document}
\maketitle

\tableofcontents

\clearpage

\section{現状の課題}

\section{課題の解決方法}

\section{機能の概要・前提条件・制約事項}

\section{情報・金銭の流れ}

\section{想定する利用者}

\section{運用・保守}

\section{ハードウェア・ソフトウェアの構成}

\section{費用・効果}
\subsection{システムの効果}
本システムを導入することによって,ドローンを用いた無人宅配が可能となり,宅配サービスの人手不足を解消することができると考えられる.また,これから先進するであろうリモート診療に利用できると考えられる.

\subsection{収益}
本システムの収益は配達による収入を想定している.配送料を600円,1日に配達する荷物の数を20,000個とすると5年間の配達による収入は
\begin{center}
    600円×20,000個×30日×60ヶ月=21,600,000,000円
\end{center}
となる.

\subsection{システムの導入・運用コスト}
本システムの導入コストは表1,運用コストは表2のようになる.
\begin{table}[htbp]
    \centering
    \begin{tabular}{c c c c c}
    \hline
    項目 & 単価(円) & 数量 & 金額(円) & 備考 \\
    \hline \hline
    物流ドローン & 3,000,000 & 200台 & 600,000,000 &  \\
    サーバ(レンタル) & 2,000 & 4ヶ月×2台 & 16,000 & \\
    \hline \hline
     & 合計 &  & 600,016,000 &  \\
    \hline
    \end{tabular}
    \caption{導入コスト}
    \label{tab:label1}
\end{table}

\begin{table}[htbp]
    \centering
    \begin{tabular}{c c c c c}
    \hline
    項目 & 単価(円) & 数量 & 金額(円) & 備考 \\
    \hline \hline
    サーバ(レンタル) & 2,000 & 60ヶ月×2台 & 240,000 & 12ヶ月×5年×1台 \\
    維持費用 & 600,016,000 & 5年 & 300,008,000 & (導入コスト×10\% )×5年 \\
    \hline \hline
     & 合計 &  & 300,248,000 &  \\
    \hline
    \end{tabular}
    \caption{運用コスト}
    \label{tab:label2}
\end{table}

上記を踏まえて,システムの開発と5年間の運用にかかる費用は次のようになる.
\begin{center}
    600,016,000円(導入コスト)+300,248,000円(運用コスト)=900,264,000円
\end{center}

\subsection{利益}
本システムを5年間運用した際の利益は以下のようになる.
\begin{center}
    21,600,000,000円(収益)-900,264,000(費用)=20,699,736,000円
\end{center}

\section{スケジュール}

\section{本システムのアピールポイント}

\section{貢献度}


\end{document}
