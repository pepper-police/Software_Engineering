\documentclass[a4paper, titlepage]{jsarticle}

\date{\today}
\usepackage[dvipdfmx]{graphicx}
\usepackage{url}
% \usepackage[T1]{fontenc}
\usepackage{float}
\usepackage{ascmac}
\usepackage{pdfpages}
\usepackage{enumitem}
\usepackage{otf}

\newcommand{\system}{\textsl{Aero Net}}

\begin{document}
\begin{titlepage}
  \centering
  \vspace*{150truept}
  {\Large 外部設計書}\\
  \vspace*{50truept}
  {\Huge ドローン宅配事業者支援システム} \\
  \vspace{15truept}
  {\Huge \system} \\
  \vspace{50truept}
  {\LARGE 土佐山田IT株式会社}\\
  \vspace{20truept}
  {\large{\tabcolsep = 1cm
      \begin{tabular}{ccc}
        久保田 天治 & 塩澤 康志 & 蝉 祐介  \\
        寺内 俊輔  & 林 晃太郎 & 松本 吏司
      \end{tabular}
    }}
\end{titlepage}

\tableofcontents

\clearpage

\section{機能の概要}

\subsection{要件定義}

\subsubsection{機能要件}
%書くべきこと
%なぜAeroNetが必要なのか、現状の課題点を説明しシステムの目的を明記する。システムの目的を達成するために必要となる機能を列挙しそれらについての軽い説明を行う

宅配業界が抱えている問題として宅配荷物数の増加と労働力不足が挙げられる.特に山間部や離島への宅配は労働力・金銭的コストがかかる.これに対し本システムは山間部や離島などの地域で効率的な宅配を行うことを目的とする.

本システムでは宅配の末端をドローンに任せる.これにより以下の効果が得られる.
%なぜこの効果が得られるのか詳しく
\begin{itemize}
	\item ドローンの自動運転による労働力不足の軽減
	\item 山間部や離島での金銭的コストの軽減
\end{itemize}

%つなぎの文章
本システムの開発にあたり以下の要件を定義する.本システムでは管理者ユーザ・事業者ユーザ・利用者ユーザに対して異なるサービスを提供する.
%とりあえず箇条書き
\paragraph{管理者ユーザ向けシステム}
管理者ユーザ向けシステムはPCのWebブラウザ上での利用を想定する.
\begin{description}[labelwidth=\linewidth]
  \setlength{\leftskip}{1em}
  \item [ログイン・ログアウト機能]管理者がログイン・ログアウトする機能.
  \item [事業者管理機能]事業者の登録情報の管理閲覧,絞り込み,請求書の送付,支払い状態の管理閲覧,貸与ドローンの管理閲覧する機能.
  \item [利用者管理機能]利用者の登録情報の管理閲覧,絞り込み,請求書の送付,支払い状態の管理閲覧する機能.
  \item [事業者情報分析機能]事業者の利用状況を絞り込みやグラフを用いて分析する機能.
  \item [利用者情報分析機能]利用者の利用状況を絞り込みやグラフを用いて分析する機能.
  \item [宅配依頼受付機能]利用者から宅配依頼を受け,事業者が宅配依頼を受けることが可能か確認し,事業者に仕事を割り振る機能.受付中,受付完了,集荷中,配送中,配送完了の状態を管理する.
  \item [宅配仕事割り振り機能]宅配依頼受付機能で呼び出す,事業者に仕事を割り振る機能.集配に向かう事業者と仲介トラックと最終的に配送する事業者の組み合わせを選ぶ.
  \item [ドローン貸与機能]事業者からドローン貸与申請を受けて貸出あるいは,申請の差し戻しを行う機能.
  \item [事業者ドローン情報管理機能]事業者の所持ドローン,貸出用ドローンの登録,個数の変更,各ドローンの個体番号と稼働時間,各ドローンの状態閲覧を行う機能.
\end{description}
\paragraph{事業者ユーザ向けシステム}
事業者ユーザ向けシステムはPCのWebブラウザ上での利用を想定する.
\begin{description}[labelwidth=\linewidth]
  \setlength{\leftskip}{1em}
  \item [ログイン・ログアウト機能] 事業者用のログイン・ログアウトする機能.
  \item [事業者登録申請機能] 事業者名,事業代表者,免許情報,口座情報,事業拠点,従業員数,電話番号,メールアドレス,施設情報,パスワードを入力して事業者登録申請を行う機能.
  \item [事業者情報編集機能] 事業者名,事業代表者,免許情報,口座情報,事業拠点,従業員数,電話番号,メールアドレス,施設情報,パスワードを編集する機能.
  \item [依頼受注判断機能] 管理者から送られる宅配依頼を承諾もしくは拒否する機能.
  \item [配達完了通知機能] 宅配が完了した場合に利用者に通知する機能.
  \item [使用ドローン登録機能] 事業者が独自に購入したドローンを登録する機能.
  \item [子アカウント発行機能] 権限を限定した一般従業員用アカウントを発行して,同一事業者内で複数人が事業を行えるようにする機能.
  \item [子アカウント管理機能] 事業者が子アカウント発行機能を用いて発行した子アカウントの管理を行う機能.
  \item [ドローン貸与申請機能] ドローン貸与の申請,ドローンの返却,ドローンの修理依頼,ドローンの機体トラブル報告をする機能.
  \item [退会機能] 事業を終了し,データ削除を申請して退会する機能.
\end{description}
\paragraph{利用者ユーザ向けシステム}
利用者ユーザ向けシステムは携帯端末のWebブラウザ上での利用を想定する.
\begin{description}[labelwidth=\linewidth]
  \setlength{\leftskip}{1em}
  \item [ログイン・ログアウト機能] 利用者用のログイン・ログアウトする機能.
        %登録情報を入れる
  \item [利用者会員登録機能] 利用者名,住所,電話番号,メールアドレス,パスワードを用いて利用者が会員登録をする機能,これは管理者の許可が必要ない.
  \item [利用者会員情報編集機能] 利用者名,住所,電話番号,メールアドレス,パスワードを編集する機能.
  \item [宅配場所登録機能] 宅配で離着陸する場所を指定して登録編集する機能,外の画像を送信して申請をする.
  \item [宅配依頼機能] 管理者に対して宅配を依頼する機能.
  \item [受け取り完了通知機能] 受け取り完了を通知する機能.
  \item [お気に入り登録機能] 配送相手をお気に入り登録する機能,これを用いて簡単に宅配を依頼する機能.
  \item [退会機能] データ削除を申請して退会する機能.
\end{description}
\subsubsection{非機能要件}

\section{業務フロー図}

\section{機能設計}

\section{ユーザインタフェース設計}

\section{非機能要件 設計}

\section{データ設計}

\section{ネットワーク設計}

\bibliographystyle{junsrt}
\bibliography{References.bib}

\end{document}
