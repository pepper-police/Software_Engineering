\documentclass[a4paper, titlepage]{jsarticle}

\date{\today}
\usepackage[dvipdfmx]{graphicx}
\usepackage{url}
% \usepackage[T1]{fontenc}
\usepackage{float}
\usepackage{ascmac}
\usepackage{pdfpages}
\usepackage{enumitem}
\usepackage{otf}

\newcommand{\system}{\textsl{Aero Net}}

\begin{document}
\begin{titlepage}
  \centering
  \vspace*{150truept}
  {\Large 外部設計書}\\
  \vspace*{50truept}
  {\Huge ドローン宅配事業者支援システム} \\
  \vspace{15truept}
  {\Huge \system} \\
  \vspace{50truept}
  {\LARGE 土佐山田IT株式会社}\\
  \vspace{20truept}
  {\large{\tabcolsep = 1cm
      \begin{tabular}{ccc}
        久保田 天治 & 塩澤 康志 & 蝉 祐介  \\
        寺内 俊輔  & 林 晃太郎 & 松本 吏司
      \end{tabular}
    }}
\end{titlepage}

\tableofcontents

\clearpage

\section{機能の概要}

\section{業務フロー図}

\section{機能設計}

\section{ユーザインタフェース設計}

\section{非機能要件 設計}

\section{データ設計}

\section{ネットワーク設計}

\section{非機能要件}
本システムが保証する非機能要件は以下の通りである.
\subsection{セキュリティ対策}
本システムでは以下のセキュリティ対策を行う.
\subsubsection{ファイアウォール}
サーバーとの通信時にAWS WAFを用いてファイアウォールを作成し不正アクセスのリスクを軽減する.
\subsubsection{権限設定}
開発・運用・保守時に適切なIAMロールを設定することで不正アクセスされた際のリスクを軽減する.
\subsubsection{通信の暗号化}
ユーザーが本システムにアクセスする際の通信をSSL/TLSを用いて暗号化することで,通信内容の盗聴のリスクを軽減する.
\subsubsection{データベースの暗号化}
データベースに保存されるインスタンスをAWS KMS keyを用いて暗号化することで,不正アクセスされた際のリスクを軽減する.
\subsubsection{パスワードのハッシュ化}
パスワード認証を行う際にパスワードをハッシュ化したものを用いることでパスワード漏洩のリスクを軽減する.
\subsubsection{DDoS攻撃対策}
DDoS攻撃の対策としてAWS Shieldを用いる.
\subsection{障害対策}
本システムでは以下の障害対策を行う.
\subsubsection{サーバーの冗長化}
データベースサーバーに対して可用性向上の為,マスター・スレーブ構成の冗長化を行う.
\subsubsection{サーバーの負荷分散}
特定のwebサーバーに対して負荷が集中することを避けるために,ロードバランサーを用い負荷の分散を行う.
\bibliographystyle{junsrt}
\bibliography{References.bib}

\end{document}
