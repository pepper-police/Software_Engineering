\documentclass[a4paper, titlepage]{jsarticle}

\date{\today}
\usepackage[dvipdfmx]{graphicx}
\usepackage{url}
% \usepackage[T1]{fontenc}
\usepackage{float}
\usepackage{ascmac}
\usepackage{pdfpages}
\usepackage{enumitem}
\usepackage{otf}

\newcommand{\system}{\textsl{Aero Net}}

\begin{document}
\begin{titlepage}
  \centering
  \vspace*{150truept}
  {\Large 外部設計書}\\
  \vspace*{50truept}
  {\Huge ドローン宅配事業者支援システム} \\
  \vspace{15truept}
  {\Huge \system} \\
  \vspace{50truept}
  {\LARGE 土佐山田IT株式会社}\\
  \vspace{20truept}
  {\large{\tabcolsep = 1cm
      \begin{tabular}{ccc}
        久保田 天治 & 塩澤 康志 & 蝉 祐介  \\
        寺内 俊輔  & 林 晃太郎 & 松本 吏司
      \end{tabular}
    }}
\end{titlepage}

\tableofcontents

\clearpage

\section{機能の概要}

\section{業務フロー図}

\section{機能設計}

\section{ユーザインタフェース設計}

\section{非機能要件 設計}

\section{データ設計}

\section{ネットワーク設計}
図\ref{fig:network}は,本システムのネットワーク構成を示したものである.
本システムは,Amazon Web Services(AWS) の Amazon Cloud Front を用いて作成する.
Amazon Virtual Private Cloud(Amazon VPC) を用い,2つのアベイラビリティゾーン間でのトラフィックの共有を行う.

インターネットゲートウェイにより,各アベイラビリティゾーンへと通信を分散させ,サーバと管理者・事業者・利用者がそれぞれ通信を行う.
各アベイラビリティゾーン内では,Amazon Elastic Compute Cloud(EC2) 及び Amazon Relational Database Service(RDS) インスタンスを構築し,EC2インスタンスがリクエストの処理及びデータベースとの通信を行う.

上記の構成要素の手前で,AWS WAF を利用しセキュリティの対策を行う.

\begin{figure}[H]
  \includegraphics[width=0.8\textwidth]{./network_img.pdf}
  \caption{ネットワーク構成図}
  \label{fig:network}
\end{figure}

\bibliographystyle{junsrt}
\bibliography{References.bib}

\end{document}
