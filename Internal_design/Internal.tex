\documentclass[a4paper, titlepage]{jsarticle}

\date{\today}
\usepackage[dvipdfmx]{graphicx}
\usepackage{url}
% \usepackage[T1]{fontenc}
\usepackage{float}
\usepackage{ascmac}
\usepackage{pdfpages}
\usepackage{enumitem}
\usepackage{otf}
\usepackage{morefloats}
% \usepackage{morefloats}
\usepackage{comment}


\newcommand{\system}{\textsl{Aero Net}}
\maxdeadcycles=10000
\begin{document}
\begin{titlepage}
  \centering
  \vspace*{150truept}
  {\Large 内部設計書}\\
  \vspace*{50truept}
  {\Huge ドローン宅配事業者支援システム} \\
  \vspace{15truept}
  {\Huge \system} \\
  \vspace{50truept}
  {\LARGE 土佐山田IT株式会社}\\
  \vspace{20truept}
  {\large{\tabcolsep = 1cm
      \begin{tabular}{ccc}
        久保田 天治 & 塩澤 康志 & 蝉 祐介  \\
        寺内 俊輔  & 林 晃太郎 & 松本 吏司
      \end{tabular}
    }}
\end{titlepage}
% \fig{filename(.pdf無し)}
% \newcommand{\fig}[1]{
%   \begin{figure}[H]
%     \centering
%     \includegraphics[width=\linewidth]{pdf/#1.pdf}
%     \caption{#1}
%     \label{fig:#1}
%   \end{figure}
% }
% \fig{filename(.pdf無し)}{caption}
\newcommand{\fig}[2]{
  \begin{figure}[H]
    \centering
    \includegraphics[width=\linewidth]{flow/#1.pdf}
    \caption{#2}
    \label{fig:#1}
  \end{figure}
  \clearpage
}
\newcommand{\ui}[2]{
  \begin{figure}[H]
    \centering
    \fbox{
      \includegraphics[width=0.65\textwidth]{ui/#1}
      }
    \caption{#2}
    \label{fig:#1}
  \end{figure}
}

\tableofcontents

\clearpage

\section{はじめに}
%何か一言.何を示す文章か.
本書では,弊社がシステム提案書で提案したドローン宅配事業者支援システムの概要を説明する.
次に,動作環境及び開発環境を示す.
最後に,本システムの機能におけるモジュール及びコンポーネントの詳細を示す.

\section{システム概要}
%これはどんなシステムか
近年,日本の宅配需要は増加しているが,労働力不足が大きな課題となっている.
道路貨物運送業の自動車運転者が減少しており,これを解決するためには効率的な配送方法が必要である.
ドローン宅配事業者支援システム\system は,ドローンを活用して宅配業界の労働力不足を解決し,多くの事業者にドローン宅配を利用してもらうための基盤サービスを提供する.
このシステムを導入することにより,今後日本国内で増加するであろう宅配便取扱個数の増加と労働力不足を宅配業界全体で乗り越えられると考える.

\section{動作環境}
アプリ,PC,Windows,性能
\section{開発環境}
使用言語,コーディング規約,ファイル構成

\section{機能一覧}
%実装したい機能の一覧
本システムには,管理者向け機能,事業者向け機能,利用者向け機能が存在しており,それぞれの機能は以下に示す.

\subsection{管理者向け機能}
管理者向け機能を以下に示す.
\begin{itemize}[labelwidth=\linewidth]
  \setlength{\leftskip}{1em}

  \item ログイン機能 %管理者がユーザ名,パスワード入力フィールドに情報を入力することでログインすることができ,
  %ログイン状態を保持できる機能.入力内容が正しい場合,ログイン状態になり,トップページに遷移する.
  \item ログアウト機能 %本システムにログインしている管理者が,ログイン状態を解除できる機能.
  %ログアウトするとログイン画面に遷移する.

  \item 事業者一覧閲覧機能 %事業者の一覧を表示閲覧する機能.この機能で表示する事業者一覧から事業者に対する各種操作を行う.
  %\item 事業者絞り込み機能 %事業者の一覧から条件により絞り込みを行う機能.
  %\item 事業者検索機能 %事業者の一覧から条件により検索を行い,検索結果を表示する機能.
  %\item 事業者情報並び替え機能 %事業者の一覧からある項目を用いて並べ替えを行う機能.
  \item 請求書送付機能 %事業者に対してメールで請求書を送付する機能.
  \item 事業者情報詳細閲覧機能 %選択した事業者の詳細情報を表示する機能.
  \item 事業者情報編集機能 %選択した事業者の情報を編集する機能.
  \item 事業者支払い情報詳細閲覧機能 %事業者の支払い情報の詳細を閲覧する機能.
  \item 事業者支払い情報詳細編集機能 %事業者の支払い情報を編集する機能.
  \item 事業者ドローン情報詳細機能 %事業者のドローン情報の詳細を閲覧する機能.
  \item 事業者ドローン情報編集機能 %事業者のドローン情報を編集する機能.

  \item 利用者一覧閲覧機能 %利用者の一覧を表示閲覧する機能.この機能で表示する利用者一覧から利用者に対する各種操作を行う.
  %\item 利用者絞り込み機能 %利用者の一覧から条件により絞り込みを行う機能.
  %\item 利用者検索機能 %利用者の一覧から条件により検索を行い,検索結果を表示する機能.
  %\item 利用者情報並び替え機能 %利用者の一覧からある項目を用いて並べ替えを行う機能.
  \item 請求書送付機能 %利用者に対してメールで請求書を送付する機能.
  \item 利用者情報詳細閲覧機能 %選択した利用者の詳細情報を表示する機能.
  \item 利用者情報編集機能 %選択した利用者の情報を編集する機能.
  \item 利用者支払い情報詳細閲覧機能 %利用者の支払い情報の詳細を閲覧する機能.
  \item 利用者支払い情報詳細編集機能 %利用者の支払い情報を編集する機能.
  
  \item 事業者統計情報表示機能 %事業者の統計情報を表示する機能.
  %\item 事業者情報絞り込み機能 %事業者の統計情報を絞り込む機能.
  %\item 事業者情報グラフ表示機能 %事業者情報のグラフを表示する機能.
  
  \item 利用者統計情報表示機能 %利用者の統計情報を表示する機能.
  %\item 利用者情報絞り込み機能 %利用者の統計情報を絞り込む機能.
  %\item 利用者情報グラフ表示機能 %利用者情報のグラフを表示する機能.
  
  \item 宅配依頼一覧表示機能 %宅配依頼一覧を表示する機能.
  %\item 絞り込み機能 %宅配依頼の絞り込みを行う機能.
  %\item 検索機能 %宅配依頼の検索を行う機能.
  %\item 情報並び替え機能 %宅配依頼一覧の並び替えを行う機能.
  
  \item 宅配仕事割り振り機能 %宅配依頼受付機能で呼び出す,事業者に仕事を割り振る機能.集配に向かう事業者と仲介トラックと最終的に配送する事業者の組み合わせを選ぶ.(基本的には自動で全部行う,動かなくなったらエラーを返す)
  %\item ドローン貸与申請一覧機能 %ドローン貸与申請の一覧を表示する機能.
  %\item ドローン貸与機能 %事業者からドローン貸与申請を受けて貸出あるいは,申請の差し戻しを行う機能.
  %\item 絞り込み機能 %宅配業務割り振り一覧の絞り込みを行う機能.
  %\item 検索機能 %宅配業務割り振りの検索を行う機能.
  %\item 宅配業務情報並び替え機能 %宅配業務情報の並び替えを行う機能.
  %\item 絞り込み機能 %事業者ドローン一覧の絞り込みを行う機能.
  %\item 検索機能 %事業者ドローンの検索を行う機能.
  %\item ドローン情報並び替え機能 %ドローン情報の並び替えを行う機能.
  \item 事業者ドローン情報編集 %ドローンの各種詳細情報を表示する機能.
  \item ドローン登録機能 %ドローンを登録する機能.
\end{itemize}

\subsection{事業者向け機能}
事業者向け機能を以下に示す.
\begin{itemize}[labelwidth=\linewidth]
  \setlength{\leftskip}{1em}

  \item ログイン機能  %事業者がユーザ名,パスワード入力フィールドに情報を入力することでログインすることができ,
  %ログイン状態を保持できる機能.入力内容が正しい場合,ログイン状態になり,トップページに遷移する.
  \item ログアウト機能  %本システムにログインしている事業者が,ログイン状態を解除できる機能.
  %ログアウトするとTOP画面に遷移する.
  %\item 事業者登録申請機能  %事業者名,事業代表者,免許情報,口座情報,事業拠点,従業員数,電話番号,メールアドレス,施設情報,パスワードを入力して事業者登録申請を行う機能.
  \item 事業者情報編集機能  %事業者名,事業代表者,免許情報,口座情報,事業拠点,従業員数,電話番号,メールアドレス,施設情報,パスワードを編集する機能.

  %\item 絞り込み機能 %配達依頼一覧の絞り込みを行う機能.
  %\item 検索機能 %配達依頼の検索を行う機能.
  %\item 配達依頼並び替え機能 %情報の並び替えを行う機能.
  \item 配達完了通知機能  %宅配が完了した場合に利用者に通知する機能.
  \item 使用ドローン登録機能  %事業者が独自に購入したドローンを登録する機能.

  \item 子アカウント一覧表示機能 %子アカウント一覧を表示する機能.
  \item 子アカウント発行機能 %権限を限定した一般従業員用アカウントを発行して,同一事業者内で複数人が事業を行えるようにする機能.
  \item 子アカウント削除機能 %子アカウントの削除を行う機能.
  \item 子アカウント編集機能 %子アカウントの編集を行う機能.

  \item ドローン種類一覧機能 %ドローンの種類を一覧表示する機能.ここから,ドローンの貸与申請を行う.
  \item ドローンの修理依頼機能 %ドローンの修理を依頼する機能.
  \item ドローンの機体トラブル報告 %ドローンの機体に関するトラブルを管理者に報告する機能.
  %\item ドローン貸与申請機能 %ドローン貸与の申請,ドローンの返却,ドローンの修理依頼,ドローンの機体トラブル報告をする機能.
  \item 退会機能 %事業を終了し,データ削除を申請して退会する機能.
  \item 所持ドローン一覧機能 %所持ドローンを一覧表示する機能.
\end{itemize}

\subsection{利用者向け機能}
利用者向け機能を以下に示す.
\begin{itemize}[labelwidth=\linewidth]
  \setlength{\leftskip}{1em}

  \item ログイン機能 %利用者がユーザ名,パスワード入力フィールドに情報を入力することでログインすることができ,
  %ログイン状態を保持できる機能.
  \item ログアウト機能 %本システムにログインしている利用者が,ログイン状態を解除できる機能.
  \item 利用者会員登録機能 %利用者名,住所,電話番号,メールアドレス,パスワードを用いて利用者が会員登録をする機能,これは管理者の許可が必要ない.
  \item 利用者会員情報編集機能 %利用者名,住所,電話番号,メールアドレス,パスワードを編集する機能.
  \item 宅配場所登録機能 %宅配で離着陸する場所を指定して登録編集する機能.外の画像を送信して申請をする.
  %登録申請をしました画面を出す.
  \item 宅配依頼機能 %管理者に対して宅配を依頼する機能.
  \item お気に入り一覧表示機能 %お気に入り登録した相手を一覧表示する機能.
  \item お気に入りからデータ参照機能 %お気に入りからデータを取得して反映する機能.
  \item 受け取り完了通知機能 %受け取り完了を通知する機能.
  \item お気に入り登録機能 %配送相手をお気に入り登録する機能,これを用いて簡単に宅配を依頼する機能.
  \item 退会機能 %データ削除を申請して退会する機能.
\end{itemize}

\section{遷移図の定義}
遷移図の見方

% 機能を実装するためのモジュール
\section{モジュール一覧}

\section{各モジュール定義}
\url{https://lms.kochi-tech.ac.jp/pluginfile.php/210601/mod_page/content/12/tkb_12_module_def.pdf}これを参考にする
+遷移図
\section{モジュールの依存関係}
同時に呼び出しが出来ない,呼び出しの限定


% 機能を実装するためのコンポーネント
\section{コンポーネント一覧}
一覧(名前と概要)
入力フォームのような小さい機能(部品)
\section{各コンポーネント定義}
\url{https://lms.kochi-tech.ac.jp/pluginfile.php/210601/mod_page/content/12/tkb_12_module_def.pdf}これを参考にする
+遷移図
\section{コンポーネントの依存関係}
呼び出しの制限


% できればやる
% \section{データベース設計}
% 何を保存するTableか.なんの属性を持つか.その概要と保存の規則.

\end{document}