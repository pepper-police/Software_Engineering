\documentclass[a4paper, titlepage]{jsarticle}

\date{\today}
\usepackage[dvipdfmx]{graphicx}
\usepackage{url}
% \usepackage[T1]{fontenc}
\usepackage{float}
\usepackage{ascmac}
\usepackage{pdfpages}
\usepackage{enumitem}
\usepackage{otf}
\usepackage{morefloats}
% \usepackage{morefloats}
\usepackage{comment}


\newcommand{\system}{\textsl{Aero Net}}
\maxdeadcycles=10000
\begin{document}
\begin{titlepage}
  \centering
  \vspace*{150truept}
  {\Large 内部設計書}\\
  \vspace*{50truept}
  {\Huge ドローン宅配事業者支援システム} \\
  \vspace{15truept}
  {\Huge \system} \\
  \vspace{50truept}
  {\LARGE 土佐山田IT株式会社}\\
  \vspace{20truept}
  {\large{\tabcolsep = 1cm
      \begin{tabular}{ccc}
        久保田 天治 & 塩澤 康志 & 蝉 祐介  \\
        寺内 俊輔  & 林 晃太郎 & 松本 吏司
      \end{tabular}
    }}
\end{titlepage}
% \fig{filename(.pdf無し)}
% \newcommand{\fig}[1]{
%   \begin{figure}[H]
%     \centering
%     \includegraphics[width=\linewidth]{pdf/#1.pdf}
%     \caption{#1}
%     \label{fig:#1}
%   \end{figure}
% }
% \fig{filename(.pdf無し)}{caption}
\newcommand{\fig}[2]{
  \begin{figure}[H]
    \centering
    \includegraphics[width=\linewidth]{flow/#1.pdf}
    \caption{#2}
    \label{fig:#1}
  \end{figure}
  \clearpage
}
\newcommand{\ui}[2]{
  \begin{figure}[H]
    \centering
    \fbox{
      \includegraphics[width=0.65\textwidth]{ui/#1}
      }
    \caption{#2}
    \label{fig:#1}
  \end{figure}
}

\tableofcontents

\clearpage

\section{はじめに}
何か一言.何を示す文章か.
\section{システム概要}
これはどんなシステムか
\section{動作環境}
アプリ,PC,Windows,性能
\section{開発環境}
使用言語,コーディング規約,ファイル構成

\section{機能一覧}
実装したい機能の一覧

\section{遷移図の定義}
遷移図の見方

% 機能を実装するためのモジュール
\section{モジュール一覧}

\section{各モジュール定義}
\url{https://lms.kochi-tech.ac.jp/pluginfile.php/210601/mod_page/content/12/tkb_12_module_def.pdf}これを参考にする
+遷移図
\section{モジュールの依存関係}
同時に呼び出しが出来ない,呼び出しの限定


% 機能を実装するためのコンポーネント
\section{コンポーネント一覧}
一覧(名前と概要)
入力フォームのような小さい機能(部品)
\section{各コンポーネント定義}
\url{https://lms.kochi-tech.ac.jp/pluginfile.php/210601/mod_page/content/12/tkb_12_module_def.pdf}これを参考にする
+遷移図
\section{コンポーネントの依存関係}
呼び出しの制限


% できればやる
% \section{データベース設計}
% 何を保存するTableか.なんの属性を持つか.その概要と保存の規則.

\end{document}