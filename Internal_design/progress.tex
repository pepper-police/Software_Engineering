\documentclass[dvipdfmx]{beamer}
\usepackage{pxjahyper, float, amsmath, latexsym, amssymb, bm, ascmac, mathtools, multicol, tcolorbox, subfig, graphicx, color, ulem, url, mathrsfs}
\renewcommand{\kanjifamilydefault}{\gtdefault} %日本語フォントをゴシックに
\usefonttheme{professionalfonts} % sansではなくLaTeX文書と同じ数式モードに

\usetheme{Madrid} % デザイン
\usecolortheme[RGB={0, 128, 0}]{structure} % 色を選択
\setbeamertemplate{navigation symbols}{} % 下のナビを消す
\setbeamertemplate{footline}[frame number] % フットの番号以外を消す

%----------------------------------------
\begin{document}

\begin{frame}{進捗報告(蝉) 12/1(金)} 
  \begin{itemize}
  \item 動作環境を管理者・事業者と利用者に分けて追加 \\
    \begin{itemize}
    \item OS, CPU, メモリの項目を追加
    \item サーバの動作環境っているのか?
    \end{itemize}
  \item 使用言語PHPとフレームワークLaravelを追加 \\
    \begin{itemize}
    \item MySQLって使う?
    \item ブラウザとかバージョン管理, エディタとかあった方がいいのか?
    \end{itemize}
  \end{itemize}
\end{frame}

\end{document}
