\documentclass[a4paper, titlepage]{jsarticle}

\date{\today}
\usepackage[dvipdfmx]{graphicx}
\usepackage{url}
% \usepackage[T1]{fontenc}
\usepackage{float}
\usepackage{ascmac}
\usepackage{pdfpages}
\usepackage{enumitem}
\usepackage{otf}

\newcommand{\system}{\textsl{Aero Net}}

\begin{document}
\begin{titlepage}
  \centering
  \vspace*{150truept}
  {\Large 外部設計書}\\
  \vspace*{50truept}
  {\Huge ドローン宅配事業者支援システム} \\
  \vspace{15truept}
  {\Huge \system} \\
  \vspace{50truept}
  {\LARGE 土佐山田IT株式会社}\\
  \vspace{20truept}
  {\large{\tabcolsep = 1cm
      \begin{tabular}{ccc}
        久保田 天治 & 塩澤 康志 & 蝉 祐介  \\
        寺内 俊輔  & 林 晃太郎 & 松本 吏司
      \end{tabular}
    }}
\end{titlepage}

\tableofcontents

\clearpage

\section{はじめに}
\subsection{システム化の背景・目的}
提案書の内容の要約を書く
\subsection{システム化の範囲}
後回し
\section{要件定義}
機能要件・非機能要件を記述する。要件を踏まえた具体的な設計は本章以降で説明する。
\subsection{機能要件}
後回し
\subsection{非機能要件}
後回し

\section{業務フロー図}
%流れ
\section{機能設計}
本システムは管理者,事業者,利用者が存在する.それぞれに向けた機能設計を記述する.
\subsubsection{管理者向け機能}
管理者用のwebページを作成し下記の機能を設ける.
\begin{description}[labelwidth=\linewidth]
  \setlength{\leftskip}{1em}
  \item [ログイン機能] 管理者がユーザ名、パスワード入力フィールドに情報を入力することでログインすることができ、
  ログイン状態を保持できる機能である。入力内容が正しい場合、ログイン状態になあり、トップページに遷移する
  \item [ログアウト機能] 本システムにログインしている管理者が、ログイン状態を解除できる機能である。
  ログアウトするとログイン画面に遷移する。
  \item [事業者管理機能]事業者の登録情報の管理閲覧,絞り込み,請求書の送付,支払い状態の管理閲覧,貸与ドローンの管理閲覧する機能.
  登録情報管理ボタン、支払い状態管理ボタン、貸与ドローン管理ボタンでそれぞれ対応したページ(登録情報詳細、支払い状態詳細、貸与ドローン詳細)に遷移する。
  左にあるチェックボックスを選択して、一括請求書の送付ボタンで一括で請求書を送付できる。各種データでソートと絞り込みを行えるようにする。
  全選択、チェックに一括請求書送付ボタン、絞り込みボタン
  チェックボックス、事業者名、事業拠点数、登録住所の都道府県、貸与ドローン数、直近7日間の宅配実績、支払い実績、請求書の送付、登録情報管理ボタン、支払い状態管理ボタン、貸与ドローン管理ボタンが並んでいる。
  一画面に収める
  (登録情報、支払い状態、貸与ドローン情報は基本的に自動で更新されるが、手動操作も可能にしている)

  \item [利用者管理機能]利用者の登録情報の管理閲覧,絞り込み,請求書の送付,支払い状態の管理閲覧する機能.
  登録情報管理ボタン、支払い状態管理ボタンでそれぞれ対応したページに遷移する。各種データでソートと絞り込みを行えるようにする。
  大体上と同じ
  (先払いを前提にすることで請求書送付機能いらない気がしてきた)

  \item [事業者情報分析機能]事業者の利用状況を絞り込みやグラフを用いて分析する機能.
  分析タブから事業者の統計情報を表示する機能である。(詳細は後日検討)

  \item [利用者情報分析機能]利用者の利用状況を絞り込みやグラフを用いて分析する機能.
  分析タブから利用者の統計情報を表示する機能である。(詳細は後日検討)

  \item [宅配依頼受付機能]利用者から宅配依頼を受け,事業者が宅配依頼を受けることが可能か確認し,事業者に仕事を割り振る機能.受付中,受付完了,集荷中,配送中,配送完了の状態を管理する.
  (基本的には自動で全部行う、動かなくなったらエラーを返す)

  \item [宅配仕事割り振り機能]宅配依頼受付機能で呼び出す,事業者に仕事を割り振る機能.集配に向かう事業者と仲介トラックと最終的に配送する事業者の組み合わせを選ぶ.
  (基本的には自動で全部行う、動かなくなったらエラーを返す)

  \item [ドローン貸与機能]事業者からドローン貸与申請を受けて貸出あるいは,申請の差し戻しを行う機能.
  各種データでソートと絞り込みを行えるようにする。
  上に、一括選択ボタン、一括貸出ボタン、一括拒否ボタン、絞り込みが並んでいる。
  チェックボックス、事業者名、要求ドローンの型番、要求個数、貸出ボタン、拒否ボタンが並んでいる。
  ボタンは確認が入る。
  \item [事業者ドローン情報管理機能]事業者の所持ドローン,貸出用ドローンの登録,個数の変更,各ドローンの個体番号と稼働時間,各ドローンの状態閲覧を行う機能.
  貸与ドローン詳細画面がこれ
  上に全部チェック、プルダウンで一括削除と一括登録の選択と一括返却依頼、チェックに一括請求書送付ボタン、ドローン登録ボタン、絞り込みボタン
  チェックボックス、型番、個体番号、稼働時間、ドローンの状況、返却依頼ボタン、削除ボタzン、返却ボタン
\end{description}

\subsubsection{事業者向け機能}
事業者用のwebページを作成し下記の機能を設ける.
\begin{description}[labelwidth=\linewidth]
  \setlength{\leftskip}{1em}
  \item [ログイン機能] 事業者がユーザ名、パスワード入力フィールドに情報を入力することでログインすることができ、
  ログイン状態を保持できる機能である。入力内容が正しい場合、ログイン状態になあり、トップページに遷移する
  \item [ログアウト機能] 本システムにログインしている事業者が、ログイン状態を解除できる機能である。
  ログアウトするとTOP画面に遷移する。
  \item [事業者登録申請機能] 事業者名,事業代表者,免許情報,口座情報,事業拠点,従業員数,電話番号,メールアドレス,施設情報,パスワードを入力して事業者登録申請を行う機能.
  \item [事業者情報編集機能] 事業者名,事業代表者,免許情報,口座情報,事業拠点,従業員数,電話番号,メールアドレス,施設情報,パスワードを編集する機能.
  \item [依頼受注判断機能] 管理者から送られる宅配依頼を承諾もしくは拒否する機能.
  上に一括チェック

  \item [配達完了通知機能] 宅配が完了した場合に利用者に通知する機能.
  \item [使用ドローン登録機能] 事業者が独自に購入したドローンを登録する機能.
  \item [子アカウント発行機能] 権限を限定した一般従業員用アカウントを発行して,同一事業者内で複数人が事業を行えるようにする機能.
  \item [子アカウント管理機能] 事業者が子アカウント発行機能を用いて発行した子アカウントの管理を行う機能.
  \item [ドローン貸与申請機能] ドローン貸与の申請,ドローンの返却,ドローンの修理依頼,ドローンの機体トラブル報告をする機能.
  \item [退会機能] 事業を終了し,データ削除を申請して退会する機能.
\end{description}

\subsubsection{利用者向け機能}
利用者向けのスマートフォン用ソフトウェアを用意し下記の機能を設ける.
\begin{description}[labelwidth=\linewidth]
  \setlength{\leftskip}{1em}
  \item [ログイン機能] 利用者がユーザ名、パスワード入力フィールドに情報を入力することでログインすることができ、
  ログイン状態を保持できる機能である。入力内容が正しい場合、ログイン状態になあり、トップページに遷移する
  \item [ログアウト機能] 本システムにログインしている利用者が、ログイン状態を解除できる機能である。
  ログアウトするとTOP画面に遷移する。
  \item [利用者会員登録機能] 利用者名,住所,電話番号,メールアドレス,パスワードを用いて利用者が会員登録をする機能,これは管理者の許可が必要ない.
  \item [利用者会員情報編集機能] 利用者名,住所,電話番号,メールアドレス,パスワードを編集する機能.
  \item [宅配場所登録機能] 宅配で離着陸する場所を指定して登録編集する機能,外の画像を送信して申請をする.
  \item [宅配依頼機能] 管理者に対して宅配を依頼する機能.
  \item [受け取り完了通知機能] 受け取り完了を通知する機能.
  \item [お気に入り登録機能] 配送相手をお気に入り登録する機能,これを用いて簡単に宅配を依頼する機能.
  \item [退会機能] データ削除を申請して退会する機能.
\end{description}

\section{ユーザインタフェース設計}

\section{運用・保守設計}
本システムは宅配依頼のAWSで構築することを前提とする、
従って24時間365日という時間はオンライン/バッチを含みシステムが稼働している時間として定義する。
ただし、毎月頭の深夜1時間程度メンテナンスの為にシステムを停止させる。
\subsection{運用設計}
ここ調べないと分からない。
基本的にはDBのみ一定時間ごとにバックアップを取る、バックアップの間の時間は変更の差分のみ保存する、各種サーバはDockerか何かですぐに立ち上がるようにする。
具体的なやつはAWSの機能を調べたら良いらしい

\subsection{保守設計}
保守設計はシステムに変更を加えてシステムを維持することを目的とする。
(先輩の資料が何を言ってるのかよく分からん)
logと管理者からの報告からバグをリストアップした後、ドキュメント化し開発環境にて修正を行う。
システムアップデートについても開発環境にて検証を行う。
バグの修正とシステムアップデートは致命的なもの以外は毎月のメンテナンスにて反映する。
緊急性の高いものは開発環境で動作確認を終えた後、迅速に緊急メンテナンスを行い反映する。
\section{データ設計}

\section{ネットワーク設計}
まだ詳細は決まってない

\bibliographystyle{junsrt}
\bibliography{References.bib}

\end{document}
